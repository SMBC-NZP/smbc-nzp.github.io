\documentclass[]{article}
\usepackage{lmodern}
\usepackage{amssymb,amsmath}
\usepackage{ifxetex,ifluatex}
\usepackage{fixltx2e} % provides \textsubscript
\ifnum 0\ifxetex 1\fi\ifluatex 1\fi=0 % if pdftex
  \usepackage[T1]{fontenc}
  \usepackage[utf8]{inputenc}
\else % if luatex or xelatex
  \ifxetex
    \usepackage{mathspec}
  \else
    \usepackage{fontspec}
  \fi
  \defaultfontfeatures{Ligatures=TeX,Scale=MatchLowercase}
\fi
% use upquote if available, for straight quotes in verbatim environments
\IfFileExists{upquote.sty}{\usepackage{upquote}}{}
% use microtype if available
\IfFileExists{microtype.sty}{%
\usepackage{microtype}
\UseMicrotypeSet[protrusion]{basicmath} % disable protrusion for tt fonts
}{}
\usepackage[margin=1in]{geometry}
\usepackage{hyperref}
\hypersetup{unicode=true,
            pdftitle={Introduction to R and R Studio},
            pdfauthor={Brian S. Evans, Ph.D.},
            pdfborder={0 0 0},
            breaklinks=true}
\urlstyle{same}  % don't use monospace font for urls
\usepackage{color}
\usepackage{fancyvrb}
\newcommand{\VerbBar}{|}
\newcommand{\VERB}{\Verb[commandchars=\\\{\}]}
\DefineVerbatimEnvironment{Highlighting}{Verbatim}{commandchars=\\\{\}}
% Add ',fontsize=\small' for more characters per line
\usepackage{framed}
\definecolor{shadecolor}{RGB}{248,248,248}
\newenvironment{Shaded}{\begin{snugshade}}{\end{snugshade}}
\newcommand{\KeywordTok}[1]{\textcolor[rgb]{0.13,0.29,0.53}{\textbf{#1}}}
\newcommand{\DataTypeTok}[1]{\textcolor[rgb]{0.13,0.29,0.53}{#1}}
\newcommand{\DecValTok}[1]{\textcolor[rgb]{0.00,0.00,0.81}{#1}}
\newcommand{\BaseNTok}[1]{\textcolor[rgb]{0.00,0.00,0.81}{#1}}
\newcommand{\FloatTok}[1]{\textcolor[rgb]{0.00,0.00,0.81}{#1}}
\newcommand{\ConstantTok}[1]{\textcolor[rgb]{0.00,0.00,0.00}{#1}}
\newcommand{\CharTok}[1]{\textcolor[rgb]{0.31,0.60,0.02}{#1}}
\newcommand{\SpecialCharTok}[1]{\textcolor[rgb]{0.00,0.00,0.00}{#1}}
\newcommand{\StringTok}[1]{\textcolor[rgb]{0.31,0.60,0.02}{#1}}
\newcommand{\VerbatimStringTok}[1]{\textcolor[rgb]{0.31,0.60,0.02}{#1}}
\newcommand{\SpecialStringTok}[1]{\textcolor[rgb]{0.31,0.60,0.02}{#1}}
\newcommand{\ImportTok}[1]{#1}
\newcommand{\CommentTok}[1]{\textcolor[rgb]{0.56,0.35,0.01}{\textit{#1}}}
\newcommand{\DocumentationTok}[1]{\textcolor[rgb]{0.56,0.35,0.01}{\textbf{\textit{#1}}}}
\newcommand{\AnnotationTok}[1]{\textcolor[rgb]{0.56,0.35,0.01}{\textbf{\textit{#1}}}}
\newcommand{\CommentVarTok}[1]{\textcolor[rgb]{0.56,0.35,0.01}{\textbf{\textit{#1}}}}
\newcommand{\OtherTok}[1]{\textcolor[rgb]{0.56,0.35,0.01}{#1}}
\newcommand{\FunctionTok}[1]{\textcolor[rgb]{0.00,0.00,0.00}{#1}}
\newcommand{\VariableTok}[1]{\textcolor[rgb]{0.00,0.00,0.00}{#1}}
\newcommand{\ControlFlowTok}[1]{\textcolor[rgb]{0.13,0.29,0.53}{\textbf{#1}}}
\newcommand{\OperatorTok}[1]{\textcolor[rgb]{0.81,0.36,0.00}{\textbf{#1}}}
\newcommand{\BuiltInTok}[1]{#1}
\newcommand{\ExtensionTok}[1]{#1}
\newcommand{\PreprocessorTok}[1]{\textcolor[rgb]{0.56,0.35,0.01}{\textit{#1}}}
\newcommand{\AttributeTok}[1]{\textcolor[rgb]{0.77,0.63,0.00}{#1}}
\newcommand{\RegionMarkerTok}[1]{#1}
\newcommand{\InformationTok}[1]{\textcolor[rgb]{0.56,0.35,0.01}{\textbf{\textit{#1}}}}
\newcommand{\WarningTok}[1]{\textcolor[rgb]{0.56,0.35,0.01}{\textbf{\textit{#1}}}}
\newcommand{\AlertTok}[1]{\textcolor[rgb]{0.94,0.16,0.16}{#1}}
\newcommand{\ErrorTok}[1]{\textcolor[rgb]{0.64,0.00,0.00}{\textbf{#1}}}
\newcommand{\NormalTok}[1]{#1}
\usepackage{graphicx,grffile}
\makeatletter
\def\maxwidth{\ifdim\Gin@nat@width>\linewidth\linewidth\else\Gin@nat@width\fi}
\def\maxheight{\ifdim\Gin@nat@height>\textheight\textheight\else\Gin@nat@height\fi}
\makeatother
% Scale images if necessary, so that they will not overflow the page
% margins by default, and it is still possible to overwrite the defaults
% using explicit options in \includegraphics[width, height, ...]{}
\setkeys{Gin}{width=\maxwidth,height=\maxheight,keepaspectratio}
\IfFileExists{parskip.sty}{%
\usepackage{parskip}
}{% else
\setlength{\parindent}{0pt}
\setlength{\parskip}{6pt plus 2pt minus 1pt}
}
\setlength{\emergencystretch}{3em}  % prevent overfull lines
\providecommand{\tightlist}{%
  \setlength{\itemsep}{0pt}\setlength{\parskip}{0pt}}
\setcounter{secnumdepth}{0}
% Redefines (sub)paragraphs to behave more like sections
\ifx\paragraph\undefined\else
\let\oldparagraph\paragraph
\renewcommand{\paragraph}[1]{\oldparagraph{#1}\mbox{}}
\fi
\ifx\subparagraph\undefined\else
\let\oldsubparagraph\subparagraph
\renewcommand{\subparagraph}[1]{\oldsubparagraph{#1}\mbox{}}
\fi

%%% Use protect on footnotes to avoid problems with footnotes in titles
\let\rmarkdownfootnote\footnote%
\def\footnote{\protect\rmarkdownfootnote}

%%% Change title format to be more compact
\usepackage{titling}

% Create subtitle command for use in maketitle
\newcommand{\subtitle}[1]{
  \posttitle{
    \begin{center}\large#1\end{center}
    }
}

\setlength{\droptitle}{-2em}

  \title{Introduction to R and R Studio}
    \pretitle{\vspace{\droptitle}\centering\huge}
  \posttitle{\par}
    \author{Brian S. Evans, Ph.D.}
    \preauthor{\centering\large\emph}
  \postauthor{\par}
    \date{}
    \predate{}\postdate{}
  

\begin{document}
\maketitle

\subsection{Introduction to R Studio}\label{introduction-to-r-studio}

This document will begin to prepare you for the material presented in
this course. Please read and complete the following steps. Even if you
are entering this class with loads of R experience, please thoroughly
read and complete each section of this document, as each section
contains terminology you may be unfamiliar with and the beginnings of
the R data science workflow that we will develop over the course of this
class.

Throughout documents of this class there are a number of icons and
formats to pay special attention to:

 Represents sections where user input is necessary.

\texttt{Gray\ boxes} are sections that appear exactly as they would in
the R environment. You may copy-and-paste these sections directly into R
(CTRL+C to copy and CTRL+V to paste).

The spy icon () provide tips and tricks for users.

The end of each document will provide summary information on keyboard
shortcuts, operators, functions, and vocabulary.

Important! Throughout this class we will be using keyboard shortcuts.
Class documents refer to Windows/Linux keyboard mapping. Mac users
should use ``command'' instead of CTRL to execute keyboard shortcuts.

\subsection{R and the R Studio
environment}\label{r-and-the-r-studio-environment}

 Our first step is to ensure that you have R and R Studio Desktop
installed on your computer.

If you do not have R installed on your computer, visit this link and
follow the directions to do so.

If you do not have R Studio Desktop installed on your computer, visit
this link and follow the directions to do so.

 Please ensure that both programs are installed on your computer and
open R Studio.

R Studio is an interface to R that has numerous advantages over using
R's built-in interface. More than anything, R Studio provides an
environment that is easy to organize. R Studio is divided into four
panels, as described in the image below. We will use panel names and
their functionality considerably throughout this class, so please take a
moment to familiarize yourself with them.

 Open R Studio and create a new R script in the source panel using the
keyboard combination CTRL+SHIFT+N. Even before you begin writing any
code, it's good to save the script you will be working on. Save the file
on your computer using the keyboard combination CTRL+S. Name the file
``rIntro'' (note: the extension, ``.R'' will be assigned automatically).

The source panel

The source panel provides a useful interface for writing and editing
scripts and is a great tool for observing and communicating what you
have done. To execute code from the source panel, place your cursor on
the line you would like to run and use the keyboard shortcut CTRL+Enter.
To run multiple lines of code at once, use your mouse to highlight the
section of code you would like to run and hit CTRL+Enter.

 Type \texttt{1+1+2} on the first line of your script in the source
panel. Use the keyboard shortcut CTRL+Enter to execute the line of code
that the cursor is on.

 Type or copy-and-paste the following into the source panel, select the
section of code with your mouse, and run the section using CTRL+Enter:

\begin{Shaded}
\begin{Highlighting}[]
\DecValTok{1}\OperatorTok{+}\DecValTok{1}\OperatorTok{+}\DecValTok{2}
\DecValTok{3}\OperatorTok{*}\DecValTok{5}\OperatorTok{*}\DecValTok{8}
\end{Highlighting}
\end{Shaded}

 You may also use your mouse and CTRL+Enter to run a subset of a line of
code. Using your entry above, highlight just \texttt{5*8} with your
mouse and hit CTRL+Enter to run.

Note: Make sure that you have CTRL+Enter in your muscle memory -- you
are going to be using it a lot!

 Any code that you will use more than once or upon which future lines of
code depend should always be written and executed in the source panel.

The console panel

The console panel is the R Studio interface to program R. When a line of
code is run from the command prompt (the symbol \texttt{\textgreater{}})
on the console, the interface sends the code to R which evaluates it and
returns any potential output. When you ran the above sections of code
from the source panel, you may have noticed that the input and output
appeared in the console panel. This highlights a component of how R
Studio works: code written in the source panel is sent to the console
panel (CTRL+Enter), the console panel sends the code to R for
evaluation, and R prints the results in the console panel (if
applicable).

You can run code directly from the console panel. To do so, simply type
the code you would like to run and press enter.

 Type \texttt{1+1+2} after the command prompt on the console panel and
press enter to run.

 Copy-and-paste the following into the console panel and press enter to
run the section:

\begin{Shaded}
\begin{Highlighting}[]
\DecValTok{1}\OperatorTok{+}\DecValTok{1}\OperatorTok{+}\DecValTok{2}
\DecValTok{3}\OperatorTok{*}\DecValTok{5}\OperatorTok{*}\DecValTok{8}
\end{Highlighting}
\end{Shaded}

 You can use your keyboard to navigate between lines of code you've run
in the console panel. This can be especially useful for modifying part
of a section of code. For example, use the up arrow on your keyboard to
change the line \texttt{1+1+2} to \texttt{1+1+2+3}.

 Any code that you will use only once should be written and executed in
the console panel. Do not write code into the console panel if future
code elements will depend on the code's output!

 Mathematical operators are similar to that in Excel (e.g., \texttt{*}
for multiplication and \texttt{/} for division). Likewise, R follows the
conventional order of operations (i.e., the same order of operations you
would find on a scientific calculator or in Microsoft Excel --
parentheses, exponents, multiplication/division, then
addition/subtraction). For example, the mathematical expression:

\[\frac{1+1+2}{3} + \frac{5}{8}\]

Would be written in R as:

\begin{Shaded}
\begin{Highlighting}[]
\NormalTok{(}\DecValTok{1}\OperatorTok{+}\DecValTok{1}\OperatorTok{+}\DecValTok{2}\NormalTok{)}\OperatorTok{/}\DecValTok{3}\OperatorTok{+}\DecValTok{5}\OperatorTok{/}\DecValTok{8} 
\end{Highlighting}
\end{Shaded}

 If you are new to R, please take a moment to explore using R as a
calculator.

The workspace panel

When you open R you initiate an R session. An R session is defined by
all of the code that you have run and any objects that you've stored in
the global environment. The global environment is a temporary storage
location for all of the objects and functions that you create in an R
working session. The workspace panel is where you can view your global
environment and your session history. In addition to viewing, you can
use this panel to save or clean your session history and objects in your
global environment. Using these tools can help streamline your R
workflow and aid in replication.

 Click on the ``History'' tab of the workspace panel to view the
commands that you have run so far.

 Click on the save icon on the history tab of the workspace panel. Save
your history as ``migBirds'' (the file type will be added
automatically). We will build on this file over the course of this
class.

 Click on the broom icon on the history tab of the workspace panel. This
will remove everything you've done in R since your session began.

Note: The workspace panel contains tools for a number of other tasks,
including tools for importing and exporting data and executing code. We
will not be using most of these features in this class but you may want
to explore them to determine whether you'd like to integrate them into
your workflow.

 Be aware of how many objects are stored in your global environment. You
can manage these objects on the ``Environment'' tab of the workspace
panel. To limit the amount of data stored in memory and make best use of
the workspace panel, be sure to remove unecessary objects.

The viewer panel

There is a lot going on in the viewer panel, but basically it provides
tools for viewing anything other than command line output. The tabs on
this panel include:

Files: A file manager window. This shows the files located in your
working directory, which is the folder on your computer that R is
reading from and writing to.

Plots: Any plots that you create can be viewed here.

Packages: R libraries that have been installed onto your system.

Help: Help files that provide in-depth information into any R function
that you will use.

Viewer: Images and maps can be viewed here.

Note: The viewer panel contains tools for a number of other tasks,
including tools for loading and installing libraries and managing files.
We will not be using most of these features in this class but you may
want to explore them to determine whether you'd like to integrate them
into your workflow.

\subsection{The assignment operator}\label{the-assignment-operator}

In R, it is often useful to save objects in your global environment by
assigning a name to them. Assignment allows you to recall saved objects
by typing the name rather than recreating the object. This is often
necessary, as objects can be quite complex.

 Type \texttt{hello\ world} into the console panel and hit CTRL+Enter.

You should have received the error message:
\texttt{Error:\ unexpected\ symbol\ in\ "hello\ world"}. If you aren't
already, you will soon become super annoyed by messages like these. So
what did you do wrong? When you type a word without quotes (which can be
single or double), R thinks you are calling an object from your
environment. The object was undefined, so it produced an error.

 Repeat the above, this time typing either
\texttt{\textquotesingle{}hello\ world\textquotesingle{}} or
\texttt{"hello\ world"} and running the line.

 Now lets define our object. We'll make an object called
\texttt{helloWorld} using R's assignment operator \texttt{\textless{}-}.
Enter or copy-and-paste the following into the source panel and execute
using CTRL+Enter:

\begin{Shaded}
\begin{Highlighting}[]
\NormalTok{helloWorld <-}
\StringTok{  'hello world'}
\end{Highlighting}
\end{Shaded}

You will notice there is no output in the console. If you look at the
workspace panel, however, you will see that your global environment now
contains a ``Value'' with the name helloWorld.

 Type \texttt{helloWorld} into the console panel and notice the printed
output.

 Enter or copy-and-paste the following into the source panel and execute
using CTRL+Enter:

\begin{Shaded}
\begin{Highlighting}[]
\NormalTok{simpleAddition <-}\StringTok{ }
\StringTok{  }\DecValTok{1} \OperatorTok{+}\StringTok{ }\DecValTok{1} \OperatorTok{+}\StringTok{ }\DecValTok{2}
\end{Highlighting}
\end{Shaded}

You have just created an object with the value of \texttt{1+1+2} and the
name \texttt{simpleAddition}. You can use the name as a representation
of the object.

 Explore the results of the object that you created by typing the
following in the console panel:

\begin{Shaded}
\begin{Highlighting}[]
\NormalTok{simpleAddition }\OperatorTok{*}\StringTok{ }\DecValTok{4}

\NormalTok{simpleAddition }\OperatorTok{+}\StringTok{ }\DecValTok{2}

\NormalTok{simpleAddition }\OperatorTok{/}\StringTok{ }\DecValTok{3}
\end{Highlighting}
\end{Shaded}

Note: If you are new to R, you probably have yet to grasp how powerful
assignments are. As most users begin to develop their R skillset, the
number of assignments they use grows. What began as a useful tool can
end up leading to an addiction that causes messy, hard-to-read scripts.
In this course you will learn how to use assignments effectively and
avoid the confusion that over-assignment causes.

 Do not use assignments unless you have to. Two key rules are:

Only define an object if you truly want to use it in the future. We
probably wouldn't really want to assign a name to \texttt{1+1+2} in the
real world.

Make sure to explore the object prior to assigning a name to it.
Assigning a name before testing an object can lead to headaches and
unseen errors in your code.

 Being careful with your naming conventions is very important! There are
a lot of conventions out there for naming objects. The most important
thing to avoid is naming an object and forgetting what you named it. The
workspace panel can help you remember names (if you keep it clean), but
the best practice is to use a consistent naming convention. Best
management practices for naming objects include:

Avoid using all caps and unnecessary capitalization. When misused,
capitalization leads to extra keystrokes without conveying additional
information.

Be careful when using punctuation to separate words as it may convey
meanings that you may not intend. For example, assigning the name
\texttt{hello.world} should be avoided because we should reserve the use
of \texttt{.} to specify file types. When using punctuation to separate
words, use \texttt{\_}. For example you may assign the name
\texttt{hello\_world} to an object.

Do use camel case! Camel case provides a clear distinction between words
while avoiding too many keystrokes or symbols that convey meaning. Camel
case is used in multi-word labels in which the first word is all
lowercase and the first letter of any additional words is uppercase. For
example, camel case would be written as camelCase.

Do not assign names to multiple versions of the same object. For example
\texttt{helloWorld1} and \texttt{helloWorld2}.

Never, never, never, never include spaces in names!

\subsection{Commenting out script:
\#comment!}\label{commenting-out-script-comment}

Good coding practices in most languages means adding plenty of
descriptive content so you (and others) know what the script is doing at
each point. This is known as adding comments or commenting your script.
Comments act as a rudder by making your scripts easy to communicate to
others, adding structure to scripts, and helping you to retrace your
steps when necessary. In R, you add a comment using \texttt{\#}. R will
not evaluate any part of a line that follows a hashtag.

 Try running the following from the source panel and observe what
happens:

\begin{Shaded}
\begin{Highlighting}[]
\CommentTok{# Adding a comment}

\DecValTok{1}\OperatorTok{+}\DecValTok{1}\OperatorTok{+}\DecValTok{2} \CommentTok{# Adding a comment}
\end{Highlighting}
\end{Shaded}

It is good to add a comment before steps in a code that describes what
happens. Make sure that the comments you've added truly describe the
process. Your comment should be able to provide a description of what a
section of code is doing that outside observers can comprehend.

 Add comments to the lines of code from the previous section:

\begin{Shaded}
\begin{Highlighting}[]
\CommentTok{# Practicing simple math and adding assignments:}
\NormalTok{simpleAddition <-}\StringTok{ }
\StringTok{  }\DecValTok{1} \OperatorTok{+}\StringTok{ }\DecValTok{1} \OperatorTok{+}\StringTok{ }\DecValTok{2}

\CommentTok{# Practicing adding an assignment to a computer-nerd phrase:}
\NormalTok{helloWorld <-}\StringTok{ }
\StringTok{  'hello world'}
\end{Highlighting}
\end{Shaded}

 Script writing is like any other type of writing in that it is best to
take your target audience into consideration when commenting a script.
If you are new to R or writing code for someone who is just starting
out, your script should be full of comments that explain every
potentially confusing process. As you develop your R skills, or are
communicating to more advanced R users, you will likely need less
commenting. We will cover best management practices in commenting
scripts considerably throughout this course.

\subsection{The function}\label{the-function}

Functions are what data-wrangling-guru Hadley Wickham calls the
``fundamental building block of R''. Functions are a special type of R
object that allow you to conduct often lengthy evaluations with limited
work. The R environment contains many functions for data management,
graphical display, and statistical analysis. To run a function, you type
the name of the function followed by a set of parentheses enclosing what
you want your function to evaluate.

 Combine the integers 1, 1, and 2 into a single object (the combine
function in R is \texttt{c}) in the source panel and name the object
\texttt{firstSet}:

\begin{Shaded}
\begin{Highlighting}[]
\CommentTok{# Combine values:}

\NormalTok{firstSet <-}\StringTok{ }
\StringTok{  }\KeywordTok{c}\NormalTok{(}\DecValTok{1}\NormalTok{, }\DecValTok{1}\NormalTok{, }\DecValTok{2}\NormalTok{)}
\end{Highlighting}
\end{Shaded}

Before you use a function it is often necessary to get a good sense of
how it works. The ``Help'' tab of the viewer panel provides in-depth
resources on each function. To access the help file for a given function
simply type \texttt{?} followed by the name of the function in the
console panel.

Functions can be used to explore, modify, or create objects as well as
define how R interacts with your computer's file system. The functions
below can be used to get the working directory R reads to and write
from, list the files in that folder, and change the working directly to
some other folder on your computer:

\begin{Shaded}
\begin{Highlighting}[]
\CommentTok{# Retrieve the location of your working directory:}

\KeywordTok{getwd}\NormalTok{()}

\CommentTok{# List the files in your current working directory:}

\KeywordTok{list.files}\NormalTok{()}

\CommentTok{# Change your working directory (modify the code below to a folder in your file system):}

\KeywordTok{setwd}\NormalTok{(}\StringTok{'C:/Users/...'}\NormalTok{)}
\end{Highlighting}
\end{Shaded}

Nested functions

Functions can be nested inside other functions. In other words, you can
run a function inside another function. This can avoid making
unnecessary assignments. To understand nested functions, it is important
to recall that the name of an object is just a simplified way to call
the object it describes.

 Enter \texttt{firstSet} and \texttt{c(1,1,2)} in the console panel
(separately) and note the output for each. Because \texttt{firstSet} IS
\texttt{c(1,1,2)}, the assigned name and the object are interchangeable.

 In the source panel, enter and run the following line of code to make a
second set of numbers with the name \texttt{secondSet}:

\begin{Shaded}
\begin{Highlighting}[]
\NormalTok{secondSet <-}\StringTok{ }
\StringTok{  }\KeywordTok{c}\NormalTok{(}\DecValTok{3}\NormalTok{, }\DecValTok{5}\NormalTok{, }\DecValTok{8}\NormalTok{)}
\end{Highlighting}
\end{Shaded}

 To explore nested functions, run the code below in the console panel,
noting the nested and non-nested versions of the function and their
output:

\begin{Shaded}
\begin{Highlighting}[]
\CommentTok{# Non-nested combination of the number sets:}

\KeywordTok{c}\NormalTok{(firstSet, secondSet)}

\CommentTok{# Combinations of the number sets using nested functions:}

\KeywordTok{c}\NormalTok{(}
  \KeywordTok{c}\NormalTok{(}\DecValTok{1}\NormalTok{, }\DecValTok{1}\NormalTok{, }\DecValTok{2}\NormalTok{),}
\NormalTok{  secondSet}
\NormalTok{)}

\KeywordTok{c}\NormalTok{(}
\NormalTok{  firstSet,}
  \KeywordTok{c}\NormalTok{(}\DecValTok{3}\NormalTok{, }\DecValTok{5}\NormalTok{, }\DecValTok{8}\NormalTok{)}
\NormalTok{)}

\KeywordTok{c}\NormalTok{(}
  \KeywordTok{c}\NormalTok{(}\DecValTok{1}\NormalTok{, }\DecValTok{1}\NormalTok{, }\DecValTok{2}\NormalTok{),}
  \KeywordTok{c}\NormalTok{(}\DecValTok{3}\NormalTok{, }\DecValTok{5}\NormalTok{, }\DecValTok{8}\NormalTok{)}
\NormalTok{)}
\end{Highlighting}
\end{Shaded}

Arguments in functions

Many functions contain arguments in addition to the object you are
evaluating. These arguments modify the behavior of the function (Note:
Some arguments are necessary while some are optional). For example, the
function \texttt{mean} contains an argument \texttt{na.rm} that tells R
whether you want to remove NA (data Not Available) prior to running the
function.

 In the source panel, make an object with two integers and an NA and
assign the name \texttt{twoValuesWithNA} to the object. In the console
panel, calculate the mean of the data with (na.rm = TRUE) and without
(na.rm = FALSE) the modifier argument:

\begin{Shaded}
\begin{Highlighting}[]
\NormalTok{twoValuesWithNA <-}\StringTok{ }
\StringTok{  }\KeywordTok{c}\NormalTok{(}\DecValTok{1}\NormalTok{, }\DecValTok{2}\NormalTok{, }\OtherTok{NA}\NormalTok{)}

\KeywordTok{mean}\NormalTok{(}
\NormalTok{  twoValuesWithNA,}
  \DataTypeTok{na.rm =} \OtherTok{FALSE}
\NormalTok{)}

\KeywordTok{mean}\NormalTok{(}
\NormalTok{  twoValuesWithNA,}
  \DataTypeTok{na.rm =} \OtherTok{TRUE}
\NormalTok{)}
\end{Highlighting}
\end{Shaded}

Libraries of functions

Base R is a set of functions that are preloaded when you downloaded R.
Many functions are available in base R but there are loads of functions
that have been written by others and assembled into libraries (also
called packages) that you can download. These user-built libraries are
very powerful and can automate many of the tasks you confront when
coding. To download a library, you use the function
\texttt{install.packages}. This function evaluates the name of a
library, in quotation marks and installs the library on your computer.
Note: Each package will only be loaded once, therefore this process
should be completed in the console panel.

 Install two libraries that we will use throughout the course of this
class:

\begin{Shaded}
\begin{Highlighting}[]
\KeywordTok{install.packages}\NormalTok{(}\StringTok{'tidyverse'}\NormalTok{)}

\KeywordTok{install.packages}\NormalTok{(}\StringTok{'RCurl'}\NormalTok{)}
\end{Highlighting}
\end{Shaded}

Once a library is installed on your computer, it is almost ready to use,
but not quite. You have to load the library into your environment using
the function \texttt{library}. Installed libraries are defined objects
therefore quotes are not necessary when loading them. Note: This
function needs to be run each time a new R session is initiated. Because
future lines of code may be dependent on the library or libraries
loaded, loading should always occur in the source panel.

 Load the tidyverse and RCurl libraries:

\begin{Shaded}
\begin{Highlighting}[]
\CommentTok{# Load libraries:}
\KeywordTok{library}\NormalTok{(tidyverse)}
\KeywordTok{library}\NormalTok{(RCurl)}
\end{Highlighting}
\end{Shaded}

 Libraries should always be loaded in the source panel! It's best to
load libraries in one of three locations:

At the top of the script

On a separate source script

Within a user-created function

 Warning! In the early stages of learning R, or even later if you have
not spent much time learning the language, you will find yourself
falling into what I call the ``library trap''. The library trap works
like this:

You need to manipulate your data in some way that you are unfamiliar
with

You visit Google or Stack overflow and, reading through the answers,
find sets of libraries that do what you want

You install and load the library and potentially move past the issue

Why is this a problem? Some of the libraries you stumble on will be
critical to your current and future workflows. Most of the libraries
found in this way will not. This turns scripts into a sloppy patchwork
of libraries. It can make code very confusing to follow and lead to
unseen errors (if the function does not do exactly what you perceive).
If the issue can be solved with Base R or the packages you are currently
working with, it will make your code more clear, help you avoid
problems, and save you time in the future. In this class we will teach
you a workflow to examine and remedy R problems using a core set of
packages.

\subsection{Term review and glossary}\label{term-review-and-glossary}

Functions and operators

\texttt{c} Combine objects

\texttt{getwd} print the location of your current working directory

\texttt{install.packages} Add a new library to a user's computer

\texttt{library} Add a new library to a user's current R session

\texttt{list.files} print the files names that are located in your
current working directory

\texttt{mean} Take the average of a set of values (note, use na.rm =
TRUE in the presence of NA's)

\texttt{setwd} Set a new location for your working directory

\texttt{\textless{}-} Assign a name to an object in R (assignment
operator)

Keyboard shortcuts

CTRL+C: Copy

CTRL+V: Paste

CTRL+S: Save file

CTRL+Shift+N: New R script

CTRL+Enter: Execute line or section of code from the source panel

Vocabulary

Argument: Condition that modifies the behavior of a function

Assign: Provide a name for an R object

Camel case: A convention of writing in which multiple words are joined
without spaces and all words following the initial word are capitalized.
forExampleThisIsCamelCase.

Code: Any text written in the source or console panel

Comment: Use a \texttt{\#} to communicate to R that the subsequent
section of code will not be evaluated

Console panel: A panel of R Studio used to execute code and
writing/running code that users do not intend to replicate

Function: Commands that can be used to execute complex or repetitive
tasks

Global environment: A storage container that provides a reference point
for all available objects in the current session

Library: A set of community-generated functions (note: Also called a
package)

Nested function: A function that is run inside of another function

Object: One of many types of data stored in the R environment, such as
individual values, functions, tables, lists, and images

Package: A set of community-generated functions (note: Also called a
library)

R session: Time spent and operations conducted since opening R (note:
sessions may be saved and loaded to avoid lost work)

Script: A file that contains all code for a set of operations

Session history: All lines of code that have been sent to R since the
beginning of a session (or since the session history was last cleared)

Source panel: A panel of R Studio used for writing, editing, and running
scripts

Viewer panel: A panel of R Studio used for navigating your system's
files, viewing help files, and image/plot output (and more)

Working directory: The location on your computer from which R reads or
writes files

Workspace panel: A panel of R Studio used to view/manage objects in your
global environment and your session history


\end{document}
